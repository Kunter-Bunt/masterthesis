\chapter{Fazit}
\label{ch:Fazit}
In dieser Arbeit wurden nach der Einleitung zunächst die Grundlagen funkbasierter Ortung und ausgewählter Funkprotokolle erörtert.
Diese waren IEEE 802.11, Bluetooth Low Energy (BLE) und Long Range (LoRa).
Anschließend wurden verwandte Arbeiten analysiert und eine Auswahl an Verfahren für die nachfolgenden Implementierungen gewählt.
Jedes der ausgewählten Protokollen wurde auf seine Reichweite im Tunnel hin untersucht.
Zusätzlich wurde für jede Implementierung der Stromverbrauch untersucht.

In dieser Arbeit hatte LoRa die höchste Reichweite im Tunnel, BLE hatte die geringste Reichweite.
Im Gegenzug hatte BLE aber den niedrigsten Stromverbrauch, IEEE 802.11 hatte bei geringerer Reichweite einen höheren Stromverbrauch als LoRa.

Diese Arbeit schlägt LoRa für \emph{zuverlässige funkbasierte Bereichsortung im Tunnelbau} vor, da die Reichweite bei hohen Sendeleistungen ausreicht, um mehrere Basisstationen zu erreichen.
Dies erlaubt eine Zuverlässigkeit bei der Ortung, die die lückenhafte Ortung mit IEEE 802.11 und BLE nicht erreichen können.

\section{Ausblick}
Neben den drei Protokollen, die in dieser Arbeit untersucht wurden, eignen sich viele weitere grundsätzlich für die Bereichsortung.
Für zukünftige Arbeiten könnten daher noch einige andere Protokolle untersucht werden.

Außerdem kann, gegeben der hohen Reichweiten von LoRa, der Umstieg von Bereichsortung auf geometrische Ortung für den Tunnelbau diskutiert werden.
Das Erreichen mehrerer Basisstationen erlaubt nun eine Triangulation der Position der mobilen Einheit.
Dazu müssten die Messgrößen von LoRa (oder einem anderen Protokoll mit sehr hoher Reichweite) auf ihre Eignung für die Lokalisierung geprüft werden und geeignete Modelle für die Signalausbreitung im Tunnel sowie der Einfluss von dynamischen Hindernissen erstellt werden.
