\chapter{Zusammenfassung und Fazit}
\label{ch:fazit}
In den letzten vier Kapiteln wurden Lösungen für die funkbasierte Bereichsortung untersucht.
Dabei wurde jeweils die Reichweite des verwendeten Protokolls geprüft, eine oder mehrere Implementierungen für die mobile Einheit erstellt und abschließend der Energieverbrauch der Lösungen bestimmt. \\

\section{Erkennungssicherheit}
Die Sicherheit bei der Erkennung von Bereichswechseln hängt maßgeblich von der Reichweite der Funktechnologie und der Länge des Sendeintervalls ab.
Tabelle \ref{table:ranges} fasst die Ergebnisse der Versuche zur Reichweite zusammen. 
Es wurden jeweils Ergebnisse ohne Gehäuse ausgewählt, da die Reichweite durch ein Gehäuse nur im Falle des ESP-12S zu einer Verringerung der Reichweite geführt hat, die Ergebnisse des ESP-12S sind jedoch zugunsten des ESP-12F nicht gelistet.

\begin{table}[h]
	\centering
	\caption{Sendereichweite mobiler Einheiten}
	\label{table:ranges}
	\begin{tabular}{p{2cm}|p{3.5cm}|p{3.5cm}|p{3cm}}
		Protokoll & Verwendetes Modul & Strecke & Maximale Sendereichweite \\
		\hline
		802.11b & ESP-12F & Wenige Hindernisse & 88m \\
		BLE 5.0 & nRF52 & Wenige Hindernisse & 32m \\
		LoRa & RFM95W 5dBm & Wenige Hindernisse & 250m \\
		LoRa & RFM95W 23dBm & Wenige Hindernisse & 1250m \\
		\hline
		802.11b & ESP-12F  & Viele Hindernisse & 32m \\
		BLE & nRF52  & Viele Hindernisse & 14m \\
		LoRa & RFM95W 5dBm & Viele Hindernisse & 100m \\
		LoRa & RFM95W 23dBm & Viele Hindernisse & >350m \\
	\end{tabular}
\end{table}

Der Abstand der Basisstationen für die Bereichsortung ist mit 250 Metern gegeben. 
Die einzige Funktechnologie, die über 125 Meter im Tunnel erreicht, ist LoRa.
Sie ist deshalb die einzige getestete Funktechnologie, die eine lückenlose Ortung ermöglicht.
Zu beachten ist, dass LoRa keine Kollisionvermeidung verwendet, stattdessen wird auf einem zufälligem Kanal gesendet, es muss davon ausgegangen werden, dass manche Nachrichten verloren gehen.
Das Sendeintervall sollte deshalb deshalb nur auf die Hälfte des gewünschten Ortungsintervalls gesetzt werden, das Sendeintervall wurde in dieser Arbeit auf zehn Sekunden gesetzt.\\
Bei Bluetooth und 802.11 entscheidet die Reichweite über das Sendeintervall, da hier Versorungslücken bei der jeweiligen Funktechnologie entstehen.
Sie werden dabei so gesetzt, dass beim Durchqueren des abgedeckten Bereichs mit 30 km/h mehrfach gesendet wird.
In dieser Arbeit wurden die Sendeintervalle für 802.11 auf fünf Sekunden und für Bluetooth Low Energy auf eine Sekunde gesetzt.\\
Kürzere Sendeintervalle führen dabei zu einer höheren Erkennungssicherheit, längere Sendeintervalle senken den Energieverbrauch.

\section{Mobile Einheiten}
Im Laufe dieser Arbeit wurden mehrere mobile Einheiten implementiert, diese sind in Tabelle \ref{table:consumptions} gelistet.

\begin{table}[h]
	\centering
	\caption{Sendereichweite mobiler Einheiten}
	\label{table:ranges}
	\begin{tabular}{p{2cm}|p{2.5cm}|p{2.5cm}|p{5.8cm}}
		Protokoll & Hardware & Art der Fernlokalisierung & Programm \\
		\hline
		802.11 & ESP8266 & Indirekt & Wifi-LLS \cite{chen2007design} \\
		802.11 & ESP8266 & Indirekt & Bereichsortung mit Assoziation \\
		\hline
		802.11 & ESP8266 & Direkt & RADAR \cite{bahl2000radar} \\
		802.11 & ESP8266 & Direkt & Ortung mit Probe Request \\
		\hline
		BLE & nRF52 & Direkt & Ortung mit Bluetooth Low Energy Advertising \cite{jianyong2014rssi} \\
		\hline
		LoRA & RFM95 & Direkt & Ortung mit LoRa RSSI \cite{kim2016poster} \\
	\end{tabular}
\end{table}

Die Implementierungen \emph{Bereichsortung mit Assoziation} und \emph{Ortung mit Probe Request} sind dabei Verbesserungen gegenüber den in der Literatur vorgeschlagenen Implementierungen, um den Energieverbrauch der jeweiligen mobilen Einheit zu senken.
Die Konzepte der anderen Implementierungen entspringen der Literatur.

\section{Energieverbrauch}
Ergänzen


\begin{table}[h]
	\centering
	\caption{Sendereichweite mobiler Einheiten}
	\label{table:ranges}
	\begin{tabular}{p{1.8cm}|p{3.5cm}|p{6cm}|p{2.3cm}}
		Protokoll & Modul & Programm  & $\varnothing$ Verbrauch\\
		\hline
		802.11 & ESP8266 Feather & Wifi-LLS & 1 \\
		802.11 & ESP-12F & Wifi-LLS & 1\\
		802.11 & ESP8266 Feather & Bereichsortung mit Assoziation & 1 \\
		802.11 & ESP-12F & Bereichsortung mit Assoziation & 1\\
		\hline
		802.11 & ESP-12F & Bereichsortung mit Assoziation (kein Access Point in Reichweite) & 1\\
		\hline
		802.11 & ESP8266 Feather & RADAR & 1\\
		802.11 & ESP-12F & RADAR & 1\\
		802.11 & ESP8266 Feather & Ortung mit Probe Request & 1\\
		802.11 & ESP-12F & Ortung mit Probe Request & 1\\
		\hline
		BLE & nRF52 Feather & Ortung mit BLE Advertising & 1\\
		BLE & nRF52 (projeziert) & Ortung mit BLE Advertising & 1\\
		\hline
		LoRa & RFM95W Feather 5dBM & Ortung mit LoRa RSSI & 1\\
		LoRa & RFM95W Feather 23dBM & Ortung mit LoRa RSSI & 1\\
	\end{tabular}
\end{table}

