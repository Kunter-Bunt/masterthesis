\chapter{Phase 3 - Mit Änderungen der Hardware}
\label{ch:phase3}
War die Infrastruktur von WLAN Access Points bisher gegeben und die Hardware somit unveränderlich, soll diese nun ausstauschbar beziehungsweise erweiterbar sein.
Dies erlaubt die Implementierung eines Bereichsortungssystems, welches nicht an die 802.11 Spezifikation gebunden ist.
Es soll eine Funkübetragungstechnik gewählt werden, die es den Tags erlaubt die in Abschnitt \ref{ch:Einleitung:sec:Anforderungen} maximal geforderten 3 Jahre Akkulaufzeit zu erreichen.
Da mehrere Topologien für das Knotennetzwerk denkbar sind, werden die Begriffe Knoten und AP im Folgenden nicht mehr gleichgesetzt.
Eine Möglichkeit wäre, APs einzusetzen, die eine zweite Funkübetragungstechnik beherrschen, optional könnnte diese Fähigkeit etwa über einen USB-Port nachgerüstet werden.
Stattdessen kann die neu eingesetzte Technik auch von der bestehenden Infrastruktur getrennt und eine neue Infrastruktur aus Knoten aufgebaut werden.
Als Kompromiss der vorherigen Möglichkeiten können sich die neuen Knoten auch mittels LAN oder WLAN in die bestehende Infrastruktur einfügen. 
Dieser Kompromiss ist grundsätzlich zu bevorzugen, da er die Komplexität geringer als bei zwei eigenständigen Netzen ist.

