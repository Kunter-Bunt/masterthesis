%TODO Bluetooth Beispiel

\chapter{Vorherige Arbeiten}
\label{ch:Vorherige}
%% ==============================
In diesem Kapitel werden vorhandene Lösungen, die für die gegebene Aufgabe der Bereichsortung in Tunneln in Frage kommen, diskutiert. 
Es werden hauptsächlich Lösungen auf Basis der 802.11 Spezifikation ausgewählt, dabei aber auch solche Systeme einbezogen, die eine spezifische Position der mobilen Einheit angeben, da diese anschließend trivial einem Bereich zugeordnet werden kann. 
Es werden sowohl wissenschaftliche als auch kommerzielle Lösungen diskutiert.\\
Auffällig ist bei der Recherche, dass wissenschaftliche Veröffentlichungen sich fast immer auf eines von zwei Aufgabenfeldern beziehen: Entweder soll die Ortungsgenauigkeit erhöht oder die Anforderungen ohne signifikanten Genauigkeitsverlust gesenkt werden. 
Diese Anforderungen können zum Beispiel Komplexität des Knoten-Netzwerks, des einzelnen Knotens, Synchronisation oder vorherige Kalibrierung sein.\\
An den Energieverbrauch der mobilen Einheiten werden üblicherweise keine Forderungen gestellt und oft von den Veröffentlichungen komplett ignoriert.
Dies ist akzeptabel wenn davon ausgegangen wird, dass die mobile Einheit zusätzlichen Nutzen bietet und das Laden der Einheit in Nutzungsszenario des Anwenders bereits vorgesehen ist.
So hat der Anwender in einem Szenario, bei dem sein Smartphone mittels direkter oder indirekter Selbstlokalisierung seine Position bestimmt um ihn zu navigieren nur minimale Anforderungen an den Energieverbrauch.
In Szenarien mit direkter oder indirekter Fernlokalisierung hat die mobile Einheit für den Anwender oft keinen direkten Nutzen, deshalb ist eine Forderung nach täglichem oder wöchentlichem Laden beziehungsweise Wechseln der Batterie schwerer durchzusetzen.\\





\subsection{Ultraschall}
Skibiniewski et al. nutzen Ultraschall für eine Fernlokalisierung mit TOF \cite{skibniewski2009simulation}.
Ultraschall kann nicht von WLAN APs empfangen werden und es muss zusätzliche Hardware installiert werden, das System soll hier dennoch Beachtung finden, da es für die Ortung von Baumaterial auf Baustellen entwickelt wurde.\\
Bei diesem System sendet zunächst der Knoten einen Beacon Frame der ZigBee Spezifikation (802.15.4) an die mobile Einheit, diese antwortet und sendet unmittelbar danach das Ultraschallsignal.
Das schnelle 2.4GHz ZigBee Signal dient dem Knoten dann als Referenz für den Sendezeitpunkt des Ultraschallsignals, welches wegen seiner langsamen Ausbreitungsgeschwindigkeit $c \approx 343m/s$ wesentlich weniger anfällig für ungenaue Zeitstempel ist. 
Außerdem kann die Zeit zwischen dem Aussenden des Beacon Frames und der Antwort gemessen werden um zusätzliche TOF-Informationen zu gewinnen. \\
Statt ZigBee ließe sich auch WLAN nutzen, ZigBee ist jedoch bereits auf niedrigen Energieverbrauch ausgelegt und wegen des Ultraschalls wäre trotzdem ein extra Knoten nötig.
Problematisch an der Lösung ist entweder der Energieverbrauch der Ultraschalleinheit oder die Reichweite, ausreichend kleine Einheiten reichen keine 100m weit und leiden trotzdem unter kurzer Batterielaufzeit.
Die Autoren schließen deshalb, dass noch Innovation in diesem Bereich notwendig ist um dieses System praktikabel zu machen.

\section{LANDMARC}
\label{ch:Vorherige:sec:LANDMARC}
Ni et al. stellen ein Ortungssystem auf Basis von Radio Frequency Identification (RFID) vor \cite{ni2004landmarc}.
Dazu werden aktive RFID-Tags als mobile Einheiten eingesetzt, diese sind mit einer Batterie ausgestattet und senden regelmäßig ihre gespeicherte ID aus.\\
Der Knoten ist mit einem Lesegerät (RFID-Reader) ausgestattet und empfängt die ID, der Knoten liefert dann die ID und ein Signalstärkelevel zwischen 1 und 8 zurück. 
Da das Signalstärkelevel nur eine sehr grobe Auflösung hat verbessern Ni et al. die Genauigkeit durch das Ausbringen von zusätzlichen Tags an bekannten, fixen Positionen, für die Bereichsortung ist dies aber nicht notwendig.\\
Die Autoren geben für ein Sendeintervall von 7,5 Sekunden eine Batterielaufzeit von 3-5 Jahren an, dabei sind die Tags klein genug und könnten, wie in der Einleitung gefordert, an einem Band um den Hals getragen werden.
%"System lässt sich leicht auf Bluetooth übertragen"

\section{Bluetooth}
\label{ch:Vorherige:sec:Bluetooth}
Hier habe ich noch kein Paper, es kommt aber fix noch mindestens ein System auf Bluetooth-Basis + Kritik an verschiedenen Buetooth Parametern.


