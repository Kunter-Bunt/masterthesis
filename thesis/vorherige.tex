\chapter{Vorherige Arbeiten}
\label{ch:Vorherige}
%% ==============================
In diesem Kapitel werden vorhandene Lösungen die für die gegebene Aufgabe der Bereichsortung in Tunneln in Frage kommen diskutiert. 
Es werden hauptsächlich Lösungen auf Basis der 802.11 Spezifikation ausgewählt, dabei aber auch solche Systeme einbezogen, die eine spezifische Position der mobilen Einheit angeben, da diese anschließend trivial einem Bereich zugeordnet werden kann. 
Es werden sowohl wissenschaftliche als auch kommerzielle Lösungen diskutiert.
Auffällig ist bei der Recherche, dass wissenschaftliche Veröffentlichungen sich fast immer auf eines von zwei Aufgabenfeldern beziehen: Entweder soll die Ortungsgenauigkeit erhöht oder die Anforderungen ohne signifikanten Genauigkeitsverlust gesenkt werden. 
Diese Anforderungen können zum Beispiel Komplexität des Knoten-Netzwerks, des einzelnen Knotens, Synchronisation oder vorherige Kalibrierung sein.\\
An den Energieverbrauch der mobilen Einheiten werden üblicherweise keine Forderungen gestellt und oft von den Veröffentlichungen komplett ignoriert.
Dies ist akzeptabel wenn davon ausgegangen wird, dass die mobile Einheit zusätzlichen Nutzen bietet und das Laden der Einheit in Nutzungsszenario des Anwenders bereits vorgesehen ist.
So hat der Anwender in einem Szenario, bei dem sein Smartphone mittels direkter oder indirekter Selbstlokalisation seine Position bestimmt um ihn zu navigieren nur minimale Anforderungen an den Energieverbrauch.
In Szenarien mit direkter oder indirekter Fernlokalisation hat die mobile Einheit für den Anwender oft keinen direkten Nutzen, deshalb ist eine Forderung nach täglichem oder wöchentlichem laden beziehungsweise wechseln der Batterie schwerer durchzusetzen.\\

\section{RADAR}
\label{ch:Vorherige:sec:RADAR}
Das RADAR System von Bahl et al. (Microsoft Research) hat als eins der ersten WLAN-basierten Ortungssysteme viel Aufmerksamkeit erfahren \cite{bahl2000radar}.
Als Messgröße wird dieStärke des empfangenen Signals (received signal strength, RSS) genutzt, diese wird laut 802.11 Spezifikation als Index (RSSI) von der Hardware zurückgegeben. 
Das RADAR System ist auf eine offline-Phase angewiesen in der empirisch ein Signalausbreitungsmodell aufgebaut wird, es handelt sich also um ein System mit Szenenanalyse.
Die Verwendung einer offline-Phase ist im stark veränderlichen Baustellenumfeld nicht akzeptabel. 
Zum einen führt der ständige Baufortschritt dazu, dass regelmäßig neu kalibriert werden muss und zum anderen wirken sich auch die großen Baumaschinen auf die Signalausbreitung aus. 
Damit sich dies nicht im Modell wiederfindet müssten zunächst alle beweglichen Maschinen aus dem Bereich entfernt werden um anschließend in der online-Phase ihren Einfluss glätten zu können.
Die offline-Phase ist deshalb wirtschaftlich gesehen nicht dürchführbar und das empirisch ermittelte Signalausbreitungsmodell müsste durch ein theoretisches ersetzt werden, für eine grobkörnige Bereichsortung sollte dies jedoch ausreichend sein.
Bei RADAR sendet die mobile Einheit 4 UDP-Pakete pro Sekunde aus, an den Knoten wird dann der RSSI gemessen, die Autoren weisen jedoch darauf hin, dass sich dieser Vorgang leicht umkehren ließe um von einer Fernlokalisation auf eine Selbstlokalisation zu kommen.
Bezüglich des Energieverbrauchs äußern sie sich jedoch zu keiner der beiden Varianten.
Die Position wird anschließend bestimmt indem aus den in der offline-Phase aufgenommenen Werten derjenige mit dem geringsten Abstand zu den gemessen Werten gewählt wird, dies wird im \textit{nearest neighbour in signal space (NNSS)} Algorithmus beschrieben.
Für die Ortung wird mehrfach gemessen und dann gemittelt um im Median eine Genauigkeit von unter 3 Metern zu erhalten, das kurze Sendeintervall von 0,25 Sekunden führt auch bei bewegten Personen zu einer Genauigkeit von 3,5 Metern.
Gleichzeitig führt das kurze Sendeintervall aber auch zu einem hohen Energieverbrauch auf Seiten der mobilen Einheit, eine Reduktion der Sendevorgänge sollte im Kontext der Bereichsotung angestrebt werden um den Energieverbrauch zu senken und die Batterielaufzeit der mobilen Einheit zu steigern.


\section{WiFi-LLS}
\label{ch:Vorherige:sec:LLS}
Chen et al. kritisieren den Einsatz von spezialisierten Access Points, diese müssen eingesetzt werden, da der RSSI nur am AP gemessen werden kann und somit vom AP für den Ortungsserver zugänglich gemacht werden muss. 
Sie wechseln daher auf eine indirekte Fernlokalisation, bei der sie die Signalstärke von Paketen naher APs wurden messen, in ein Paket packen und an den Ortungsserver senden. 
Dadurch lösen sie auch das Addressierungsproblem: RADAR stellt ursprünglich keinen Addressierungsmechanismus für mehrere mobile Einheiten zur Verfügung, vor dem Versenden des Pakets mit den Signalstärken kann nun leicht ein Identifikator wie etwa die MAC-Adresse hinzugefügt werden.
Außerdem verwenden für ihr WiFi-based Loacl Location System (WiFi-LLS) ein theoretisches Signalausbreitungsmodell $P(d) = P(d_0) - 10log_{10}(\frac{d}{d_0})^n - OAF$ mit der Distanz $d$, der Signalstärke $P(d)$ und der Referenzdistanz $d_0 = 1m$. 
$P(d_0)$, der Pfadverlustexponent $n$ und der Hindernisdämpfungsfaktor $OAF$ müssen bestimmt werden, jedoch lassen sich $P(d_0)$ und $n$ auf einer einzelnen Teststrecke mit unterschiedlichen Abständen von AP und mobiler Einheit bestimmen und $OAF$ kann für einen Gebäudetyp bestimmt werden.
Dadurch hat das Modell einen konstanten Aufwand, dies ist für Baustellen interessant, da sich diese Werte einmalig messen und dann sogar über mehrere gleichartige Baustellen übertragen ließen.
Auch in dieser Veröffentlichung steht die Ortungsgenauigkeit im Vordergrund und es werden keine Angaben zum Energieverbrauch gemacht. 
Als Referenz kann dienen, dass die mobile Einheit bei WiFi-LLS alle 5 Sekunden einen Scan (siehe Kapitel 3) durchführt, die Signalstärken der 3 signalstärksten APs zusammen mit der eigenen MAC-Adresse in XML codiert und das so erzeugte Paket an den Ortungsserver versendet.


