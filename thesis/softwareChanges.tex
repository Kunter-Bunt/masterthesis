\chapter{Direkte Fernlokalisierung mit 802.11}
\label{ch:phase2}
Alle bisher betrachteten Lösungen arbeiten mindestens auf Protokollebene vier des OSI-Modells.
Die Lösungen benutzen TCP oder UDP, um ihre Position dem Ortungsserver mitzuteilen und müsseen somit dem Netzwerk beitreten, da es sonst nicht ist mit dem Ortungsserver zu kommunizieren.\\
Da nun die Software der Access Points veränderbar ist, ergeben sich neue Freiheiten bezüglich der Kommunikation.
Weil die Zielsetzung in Abschnitt \ref{ch:Einleitung:sec:Zielsetzung} jedoch die Einschränkung macht, dass ohne Authorisierung nicht mit dem Ortungsserver kommuniziert werden darf, können die in Kapitel \ref{ch:phase1} vorgestellten Verfahren der indirekten Fernlokalisierung nicht von diesen Freiheiten Gebrauch machen.\\
Stattdessen werden in diesem Kapitel Verfahren der direkten Fernlokalisierung betrachtet.
Die Veränderbarkeit der Software des AP wird dazu genutzt auf diesem Messgrößen zu ermitteln und dann über eine Datenverbindung an den Ortungsserver zu übermitteln.
Da für die APs keine Beschränkungen bezüglich des Energieverbrauchs vorliegen, kann die Verbindung zum Ortungsserver regulär auf Schicht vier oder fünf aufgebaut werden.
Für die Kommunikation zwischen Tag und AP sollte jedoch auf niedrigere Protokollebenen ausgewichen werden, um den Energieverbrauch des Tags zu senken. \\


\section{RADAR Implementierung}
Ein RADAR Tag versendet alle 0,25 Sekunden ein 6 Byte langes UDP-Paket, der RSSI der Übertragung wird dann auf dem AP gemessen.
Das Sendeintervall wurde so kurz gewählt, um spontane Schwankungen im RSSI durch mehrfache Messung zu glätten und sich bewegende Personen möglichst genau zu erfassen.
Für eine Bereichsortung reicht ein wesentlich längeres Sendeintervall, es wird erneut ein Intervall von 5 Sekunden gewählt, in dem sich ein Mitarbeiter maximal 42 Meter bewegt. 
Tabelle \ref{table:radarconsumption} zeigt den gemessenen Verbrauch der Implementierung für RADAR Tags jeweils in Arduino und C, mit unterschiedlich langen Sendeintervallen mit und ohne manuellen \texttt{light\_sleep}.

\begin{table}[h]
	\centering
	\caption{Energieverbrauch RADAR-artiger Tags}
	\label{table:radarconsumption}
	\begin{tabular}{p{3cm}|p{2.2cm}|p{1.5cm}|p{2cm}|p{2cm}|p{2cm}}
		SDK & manueller \texttt{light\_sleep} & Sende"-intervall in s & Spannung in V & $\varnothing$ Verbrauch in mA & $\varnothing$ Verbrauch in mW \\
		\hline
		Arduino Core & Nein & 0,25 & 5 & 40 & 200 \\
		Arduino Core & Ja & 0,25 & 5 & 40 & 200 \\
		Arduino Core & Nein & 5 & 5 & 8 & 40 \\
		Arduino Core & Ja & 5 & 5 & 8 & 40 \\
		ESP Open SDK & Nein & 0,25 & 5 & 40 & 200 \\
		ESP Open SDK & Ja & 0,25 & 5 & 38 & 190 \\
		ESP Open SDK & Nein & 5 & 5 & 6 & 30 \\
		ESP Open SDK & Ja & 5 & 5 & 5 & 25 \\
	\end{tabular}
\end{table}

Der Energieverbrauch einer Lösung die vier Pakete pro Sekunde sendet ist wie erwartet hoch.
Die Implementierungen mit der ESP Open SDK und manuellem \texttt{light\_sleep} unterscheiden sich ausschließlich im Sendeintervall, dies verändert die projezierte Laufzeit jedoch stark.
Für den 1400 mAh Akku ergibt sich eine Laufzeit von 1400 mAh/38 mA $\approx$ 38,84 h für ein Sendeintervall von 0,25 Sekunden, hingegegen ergibt sich für ein Sendeintervall von 5 Sekunden eine Laufzeit von 1400\ mAh/5\ mA = 280\ h.
Das Tag mit dem längeren Sendeintervall hält somit 7,2 mal so lang wie das andere mit einer Akkuladung durch. 
Dennoch ist der Verbrauch nur marginal geringer als der der Bereichsortungstags (5,5 mA) aus Abschnitt \ref{ch:phase1:sec:anpassungbereich}. \\
Die dort besprochenen Optimierungen (nur bei AP-Wechsel senden) können auch für die RADAR Implementierung verwendet werden, das RADAR-artige Tag ist dann aber in seiner Implementierung bis auf den Inhalt des UDP-Pakets identisch mit dem Bereichsortungstag.
Der Energieverbrauch sollte sich somit kaum unterscheiden und ein System mit RADAR-artigen Tags benötigt Veränderungen der Software der APs, ein System mit Bereichortungstags ist deshalb vorzuziehen. \\
%TODO Einfach nur eingebucht bleiben

\section{Time-of-Flight Implementierung}
[Kommt noch Text, wird nicht implementiert, weil schon theoretisch teurer als RADAR]

\section{Anpassungen für Bereichsortung}
\label{ch:phase2:sec:anpassungbereich}
RADAR versendet immer noch UDP Pakete und arbeitet damit auf Schicht vier (Transport) des OSI-Modells und muss im Netzwerk authentifiziert und mit einem Access Point assoziiert sein.
Das ist für eine direkte Fernlokalisierung aber nicht notwendig, der RSSI wird auf Schicht eins (PHY) gemessen.
Grundsätzlich kann aufgrund der möglichen Änderungen am AP ein beliebiges Paket mit einer speziellen Kennung versendet und vom AP als Positionsmitteilung des Tags erkannt werden. \\
Ein Sendevorgang, der nur Schicht eins nutzt hat einen geringeren Energieverbrauch, da er nur senden und nie empfangen muss.
Ein solcher Sendevorgang könnte aber die Funktion des Netzwerks beeinträchtigen und stellt nicht sicher, dass die eigene Übertragung nicht durch andere Übertragungen gestört wurde.
Es sollte deshalb nicht auf Schicht eins gearbeitet werden.\\
Stattdessen sollte Schicht zwei (MAC) des OSI-Modells verwendet werden. 
Da 802.11 für den Mediumszugriff eine Kollisionsvermeidung (CSMA/CA) verwendet wird, muss das Tag vor dem Senden das Medium belauschen, um zu bestimmen ob es belegt ist.
Der Energieverbrauch ist somit pro Sendevorgang höher als bei einer Lösung auf Schicht eins, stellt dafür aber die Verfügbarkeit des Mediums (der Frequenz) für die übrigen Teilnehmer sicher. \\
Um die Änderungen an der Software des AP jedoch gering zu halten wurde der Probe Request als zu sendenden Frame gewählt.
Es handelt sich dabei um einen Management Frame (siehe Tabelle \ref{table:management}) der für den, in Abschnitt \ref{ch:phase1:sec:scan} beschriebenen, Scan Vorgang verwendet wird.
Der Probe Request hat dabei den Vorteil, dass er bereits vom AP verarbeitet und mit einer Probe Response beantwortet wird. \\
Es wird also lediglich gefordert, dass der AP den Empfang des Probe Request im Zuge der Verarbeitung protokolliert. 
Im Einzelnen müssen die Empfangszeit, der RSSI und die MAC-Adresse des Absenders protokolliert und für den Ortungsserver abrufbar gemacht werden. 
Manche kommerzielle APs bieten ein solches Protokoll für Probe Requests und Beacons im Zuge einer \textit{Rogue Client/AP Detection} an \cite{lancom2017rouge}.\\
Die ESP Open SDK bietet über die Operationen wie Scan und Join hinaus mit \texttt{wifi\_send\_pkt\_freedom} eine Funktion zum Senden von Paketen auf Schicht zwei an.
Der ESP8266 Arduino Core implementiert diese Funktion nicht, stattdessen muss sie mit \texttt{extern \dq C\dq $\lbrace$\#include \dq user\_interface.h\dq $\rbrace$} importiert werden. \\
\texttt{wifi\_send\_pkt\_freedom} setzt den PHY-Header selbst, der MAC-Header und Inhalt des Paketes müssen über einen Puffer übergeben werden.
\begin{verbatim}
uint8_t packet[26] = { 
/*0*/ 	0x40, //Version (2bit), Type (2bit), Subtype(4bit)
/*1*/ 	0x00, //Flags 
/*2*/ 	0x00, 0x00, //Duration
/*4*/   0xff, 0xff, 0xff, 0xff, 0xff, 0xff, //Destination MAC
/*10*/  0x01, 0x02, 0x03, 0x04, 0x05, 0x06, //Source MAC
/*16*/  0xff, 0xff, 0xff, 0xff, 0xff, 0xff, //BSSID, all ff=broadcast
/*22*/  0x00, 0x00, //Sequence Number (12bit), Fragment Number (4bit) 
//[End of MAC-Header][Start of Management tags]
/*24*/  0x83, //Tag Number (Path Reply 131) 
/*25*/ 	0x00, //Tag length
}; 
\end{verbatim}
Ein gewöhnlicher Probe Request beinhaltet noch zusätzliche Informationen bezüglich seiner technischen Möglichenkeiten, wie etwa unterstützte Standards und Datenraten. 
Da das Tag aber nicht tatsächlich beitreten will, kann darauf verzichtet werden. \\
Da keine Verbindung mehr aufrecht erhalten werden muss, können tiefere Schlafzustände eingenommen werden. 
Statt des \texttt{light\_sleep} kann der \texttt{deep\_sleep} verwendet werden.
Dieser schaltet den ESP und seinen Speicher fast vollständig ab, nach Ablauf der angegebenen Schlafzeit wird Pin 16 mit der negativen Seite der Stromquelle verbunden.
Damit der ESP wieder aufwacht muss Pin 16 mit dem Reset Pin (RST) verbunden werden, bei einem Reset initialisiert der ESP neu.
Bei einer Lösung auf einer höheren Schicht würde dies dazu führen, dass das Tag versucht dem Netztwerk erneut beizutreten. 
Hingegen kann bei einer Lösung auf Schicht zwei sofort gesendet und danach wieder geschlafen werden.\\
Tabelle \ref{table:probeconsumption} zeigt den Energieverbrauch der Implementierungen jeweils mit manuell herbeigeführtem Schlafzustand, das Sendeintervall liegt bei konstant 5 Sekunden.
\begin{table}[h]
	\centering
	\caption{Energieverbrauch Probe Request Tags}
	\label{table:probeconsumption}
	\begin{tabular}{p{3cm}|p{2.4cm}|p{2cm}|p{2cm}|p{2cm}}
		SDK & manueller Schlafzustand  & Spannung in V & $\varnothing$ Verbrauch in mA & $\varnothing$ Verbrauch in mW \\
		\hline
		Arduino Core & Ohne & 5 & 40,5 & 202,5 \\
		Arduino Core & light\_sleep & 5 & 5 & 25 \\
		Arduino Core & deep\_sleep & 5 & 4,25 & 21,25 \\
		ESP Open SDK & Ohne & 5 & 8,33 & 41,66 \\
		ESP Open SDK & light\_sleep & 5 & 3 & 15 \\
		ESP Open SDK & deep\_sleep & 5 & 0,8 & 4 \\
	\end{tabular}
\end{table}

%TODO Kann auch für genaue RSSI Trilateration genutzt werden.

Zu erkennen ist, dass die Verwendung des \texttt{deep\_sleep} zu einem geringeren Verbrauch führt. 
Allerdings benötigt das mit dem Arduino Core programmierte Tag im Vergleich zu dem mit der ESP Open SDK programmierten Tag deutlich länger zum Starten.
Es verbraucht deshalb sogar mehr Energie als die Bereichsortungstags aus Abschnitt \ref{ch:phase1:sec:anpassungbereich}, die Implementierung mit der ESP Open SDK verbraucht aber weniger Energie als diese.
Für den 1400\ mAh Akku ergibt sich eine Laufzeit von 1400\ mAh/0,8\ mA = 1750\ h für die Implementierung in C mit \texttt{deep\_sleep}.
Dies entspricht einer Laufzeit von fast 73 Tagen.
Hinzu kommt, dass sich der Verbrauch dieser Tags nicht durch die Bewegung des Trägers erhöht, da keine Reassoziationen stattfinden.
Um die in Abschnitt \ref{ch:Einleitung:sec:Anforderungen} geforderten sechs Monate Laufzeit zu erreichen, werden in Abschnitt \ref{kommtnoch} weitere Verbesserungen besprochen.

\subsection{Anzahl der verwendeten Kanäle}
Eine Lösung die nicht vor dem Versenden das Spektrum nach den Access Points durchsucht muss entweder auf der Annahme beruhen, dass alle Access Points auf einem Kanal agieren oder in allen in Frage kommenden Kanälen senden.\\
Je nach Erweiterung der 802.11 Spezifiktion ergeben sich unterschiedlich viele solcher Kanäle.
802.11b verwendet eine Kanalbreite von 22MHz, es stehen daher effektiv nur drei Kanäle zur Verfügung: 1, 7 und 13 in Europa beziehungsweise 1, 6 und 11 in Nordamerika.
Für 802.11g/n mit 20MHz Kanalbreite sind zwar in Europa theoretisch vier Kanäle verfügbar (1,5,9,13), es werden aber in der Paxis meist dieselben Kanäle wie bei 802.11b verwendet um die Kompatibilität zu 802.11b zu gewährleisten.
802.11n ist auch für eine Kanalbreite von 40 MHz spezifiziert, hier stehen effektiv nur noch 2 Kanäle zur Verfügung üblicherweise werden Kanal 3 und 11 gewählt.\\
Die Implementierung in C mit \texttt{deep\_sleep} wurde sowohl auf einem Kanal, als auch auf vier Kanälen getestet.
Dabei ergab sich in 24 Stunden kein messbarer Unterschied.
Es sollte daher auf eine Festlegung des Kanals verzichtet werden, weil diese die reguläre Funktionsweise des WLAN Netzwerks beeinträchtigen könnte.

