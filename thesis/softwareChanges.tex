\chapter{Phase 2 - Mit Softwareänderungen an den APs}
\label{ch:phase2}
Alle bisher betrachteten Lösungen arbeiteten mindestens auf Protokollebene 4 des OSI-Modells.
Die Lösungen benutzten TCP oder UDP um ihre Position dem Ortungsserver mitzuteilen und mussten somit dem Netzwerk beitreten, da es sonst nicht möglich gewesen wäre mit dem Ortungsserver zu kommunizieren.\\
Da nun die Software der Access Points veränderlich ist ergeben sich neue Freiheiten bezüglich der Kommunikation.
Da die Zielsetzung in Abschnitt \ref{ch:Einleitung:sec:Zielsetzung} jedoch die Einschränkung macht, dass ohne Authorisierung nicht mit dem Ortungsserver kommuniziert werden darf können die in Kapitel \ref{ch:phase1} vorgestellten Verfahren der indirekten Fernlokalisierung nicht von diesen Freiheiten Gebrauch machen.\\
Stattdessen werden in diesem Kapitel Verfahren der direkten Fernloaklisierung betrachtet.
Die Veränderbarkeit der Software des AP wird dazu genutzt auf diesem Messgrößen zu ermitteln und dann über eine Datenverbindung an den Ortungsserver übermittelt.
Da für die APs keine Beschränkungen bezüglich des Energieverbrauchs vorliegen, kann die Verbindung zu Ortungsserver regulär auf Schicht 4 oder 5 aufgebaut werden, für die Kommunikation zwischen Tag und AP sollte jedoch auf niedrigere Protokollebenen ausgewichen werden, um den Energieverbrauch des Tags zu senken. \\


