%% LaTeX-Beamer template for KIT design
%% by Erik Burger, Christian Hammer
%% title picture by Klaus Krogmann
%%
%% version 2.1
%%
%% mostly compatible to KIT corporate design v2.0
%% http://intranet.kit.edu/gestaltungsrichtlinien.php
%%
%% Problems, bugs and comments to
%% burger@kit.edu

\documentclass[18pt]{beamer}

%% SLIDE FORMAT

% use 'beamerthemekit' for standard 4:3 ratio
% for widescreen slides (16:9), use 'beamerthemekitwide'

\usepackage{templates/beamerthemekit}
%\usepackage{templates/beamerthemekitwide}

%% TITLE PICTURE

% if a custom picture is to be used on the title page, copy it into the 'logos'
% directory, in the line below, replace 'mypicture' with the 
% filename (without extension) and uncomment the following line
% (picture proportions: 63 : 20 for standard, 169 : 40 for wide
% *.eps format if you use latex+dvips+ps2pdf, 
% *.jpg/*.png/*.pdf if you use pdflatex)

%\titleimage{mypicture}

%% TITLE LOGO

% for a custom logo on the front page, copy your file into the 'logos'
% directory, insert the filename in the line below and uncomment it

%\titlelogo{mylogo}

% (*.eps format if you use latex+dvips+ps2pdf,
% *.jpg/*.png/*.pdf if you use pdflatex)

%% TikZ INTEGRATION

% use these packages for PCM symbols and UML classes
% \usepackage{templates/tikzkit}
% \usepackage{templates/tikzuml}

% the presentation starts here

\title[Funkbasierte Bereichsortung]{Zuverlässige funkbasierte
Bereichsortung im Tunnelbau}
\subtitle{Masterarbeit von Marius Wodtke}
\author{Marius Wodtke}

\institute{Instuitut für angewandte Informatik und Formale Beschreibungsverfahren}

% Bibliography

\usepackage[citestyle=authoryear,bibstyle=numeric,hyperref,backend=biber]{biblatex}
\addbibresource{templates/example.bib}
\bibhang1em

\begin{document}

% change the following line to "ngerman" for German style date and logos
\selectlanguage{ngerman}

%title page
\begin{frame}
\titlepage
\end{frame}

%table of contents
\begin{frame}{Gliederung}
\tableofcontents
\end{frame}

\section{Motivation}
\begin{frame}{Bisherige Situation}
\includegraphics[width=\textwidth]{images/bisherige.png}\\
\cite{maurer2016unterstuetzung}
\end{frame}

\begin{frame}{Zukünftige Situation}
\includegraphics[width=\textwidth]{images/zukuenftige.png}
\end{frame}

\begin{frame}{Aufgabe}
\begin{block}{Zielsetzung}
\begin{itemize}
\item Funkbasiertes Ortungssystem
\item Bereichsortung (250m Abschnitte)
\end{itemize}
\end{block}

\begin{block}{Anforderungen}
\begin{itemize}
\item Nichtintrusiv 
\item Zuverlässige Erkennung von Abschnittswechseln
\item Wenig Interaktion mit mobiler Einheit
\end{itemize}
\end{block}
\end{frame}

\begin{frame}{Topologien}
	\begin{tabular}{c|c|c}
		\includegraphics[width=0.3\textwidth]{images/direkteselbst.eps} & \includegraphics[width=0.3\textwidth]{images/direktefern.eps} & \includegraphics[width=0.2\textwidth]{images/ohnebasis.eps}\\
		\hline
		\includegraphics[width=0.3\textwidth]{images/indirekteselbst.eps} & \includegraphics[width=0.3\textwidth]{images/indirektefern.eps} & \includegraphics[width=0.3\textwidth]{images/hybrid.eps} \\
	\end{tabular}
\end{frame}

\begin{frame}
\begin{block}{Messgrößen}
\begin{itemize}
\item Time of Arrival
\item Time Difference of Arrival
\item Roundtrip Time of Flight
\item Received Signal Strength (Indicator)
\end{itemize}
\end{block}  
\includegraphics[width=0.3\textwidth]{images/tdoa.png} \\


\begin{block}{Protokolle}
\begin{itemize}
\item 802.11
\item Zuverlässige Erkennung von Abschnittswechseln
\item Wenig Interaktion mit mobiler Einheit
\end{itemize}
\end{block} 
\end{frame}

\appendix
\beginbackup

\begin{frame}[allowframebreaks]{References}
\printbibliography
\end{frame}

\backupend

\end{document}
